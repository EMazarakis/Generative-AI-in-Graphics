%%%%%%%%%%%%%%%%%%%%%%%%%%%%%%%%%%%%%%%%%%%%%%%%%%%%%%%%%%%%%%%%%%%%%%%%%
%
% Abstract page Greek version
%
%%%%%%%%%%%%%%%%%%%%%%%%%%%%%%%%%%%%%%%%%%%%%%%%%%%%%%%%%%%%%%%%%%%%%%%%%

\thispagestyle{empty}
\chapter*{ {ΠΕΡΙΛΗΨΗ} }
%\setmainlanguage{english}

%Για να μην έχει εσοχή η κάθε παραγράφος 
\setlength{\parindent}{0pt}	

%\emergencystretch is a TeX-internal “dimen” register, and can be set as normal for dimens in Plain TeX; in LaTeX
\setlength{\emergencystretch}{3em} 

H πτυχιακή εργασία εμβαθύνει στον τομέα αιχμής της Generative Artificial Intelligence (AI) εστιάζοντας στις εφαρμογές των γραφικών. 
Η μελέτη διερευνά διάφορες deep learning τεχνικές, συμπεριλαμβανομένων των Variational Autoencoders (VAEs), των Generative Adversarial Networks (GANs), των Diffusion Models και των Transformers. Αυτές οι μέθοδοι έχουν φέρει επανάσταση στον τομέα της Τεχνητής Νοημοσύνης, επιτρέποντας τη δημιουργία ρεαλιστικού και υψηλής ποιότητας περιεχομένου, που κυμαίνεται από εικόνες έως ολόκληρες σκηνές.\\

Η διπλωματική εργασία ξεκινά με μια εις βάθος εξερεύνηση των Variational Autoencoder, δίνοντας έμφαση στον ρόλο τους στην  λανθάνουσα αναπαράσταση των δεδομένων. Στη συνέχεια, η έρευνα μεταβαίνει στα Generative Adversarial Networks, αναδεικνύοντας τη διαδικασία εκπαίδευσης των συστατικών τους στοιχείων για τη δημιουργία αυθεντικού περιεχομένου. Επιπλέον, η μελέτη εξετάζει τα Diffusion Models, τα οποία διαπρέπουν στην probabilistic generative μοντελοποίηση, και τους Transformers , γνωστούς για την επιτυχία τους σε εργασίες παραγωγής διαδοχικών δεδομένων (sequential data).\\

Ένα σημαντικό μέρος της διπλωματικής εργασίας είναι αφιερωμένο στην εκτέλεση τριών μοντέλων στον τομέα των γραφικών με βάση ορισμένες παραμέτρους. Η εκτέλεση των μοντέλων μέσω κώδικα Python ή εκτελώντας τα τοπικά μέσω μιας διεπαφής   δίνει τη δυνατότητα στους χρήστες να αξιοποιήσουν προηγμένες τεχνικές AI. Αυτό επιτρέπει τη δημιουργία ποικίλου περιεχομένου υψηλής ποιότητας, φέρνοντας επανάσταση στις εφαρμογές σε διάφορους τομείς απρόσκοπτα.\\

Τα ευρήματα υπογραμμίζουν την επίδραση αυτών των παραγωγικών μοντέλων στο πεδίο των γραφικών, επιδεικνύοντας την ικανότητά τους να δημιουργούν πρωτότυπο και νέο οπτικό περιεχόμενο μέσω της συγχώνευσης εξελιγμένων τεχνικών βαθιάς μάθησης.

  
\vfill

{\bfseries ΘΕΜΑΤΙΚΗ ΠΕΡΙΟΧΗ}: Γενεσιουργός Τεχνητή Νοημοσύνη \\

{\bfseries ΛΕΞΕΙΣ ΚΛΙΕΔΙΑ}: Γενεσιουργός Τεχνητή Νοημοσύνη, Τεχνητή Νοημοσύνη, Μοντέλα, Γραφικά, Βαθιά Μάθηση


%%%%%%%%%%%%%%%%%%%%%%%%%%%%%%%%%%%%%%%%%%

