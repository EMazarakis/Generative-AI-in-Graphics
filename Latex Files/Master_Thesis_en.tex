%%%%%%%%%%%%%%%%%%%%%%%%%%%%%%%%%%%%%%%%%%%
%%File: 		Master_Thesis
%%Author: 	Eugene Mazarakis
%%Date:		3/12/2023
%%%%%%%%%%%%%%%%%%%%%%%%%%%%%%%%%%%%%%%%%%%

% Preamble, contains commands that affect the entire document.
\documentclass[ 11pt,a4paper,en, oneside, openright]{book} %book


\usepackage{polyglossia}

\setmainlanguage{english}
\setotherlanguage{greek}
\newcommand{\en}[1]{\begin{english}#1\end{english}}

%
% Add Greek support
\usepackage{fontspec}
\usepackage{xunicode}
\usepackage{xltxtra}

%
% Set font family Arial
\setmainfont[Ligatures={Common,TeX}, Numbers={OldStyle}]{Arial} %Times New Roman, Verdana,Courier,



%=====================================================
%=========  Required packages and configuration ====================
%=====================================================

% Define Configurations for the text inside the document
\usepackage[a4paper,
top=2cm,					%Distance of the Chapter for the top of the page
bottom=2cm,
bindingoffset=0.5cm,
left=2cm,
right=2cm,
headheight=15pt,
headsep=1.25cm,
footskip=1.25cm,
includehead,
includefoot]{geometry} % add option showframe=true for debugging


% We need to use the graphicx package, in order to tell Latex how to manage the Images.
\usepackage{graphicx}

% Scans for figures relative to this file in the dir Figures under the directory of the main document.
%\graphicspath{ {./Figures/} 

% Adding colors to the text. Making TODO red in the doc  \textcolor{red}{TODO} 
\usepackage{xcolor}

% needed for formatting chapter headings
\usepackage{sectsty}   

% needed for forcing capital letters (in chapters)
\usepackage{textcase}  

%\usepackage[resetlabels]{multibib}

% needed to adjust Toc margins and fonts
\usepackage[titles]{tocloft}  

% Chapter, Section and Subsection Style  
\usepackage{titlesec}

% Definition of chapter format
\titleformat{\chapter}{\bfseries\large\centering}{\thechapter. }{2pt}{\large}

%Definition of section format
%\titleformat{\section}{\bfseries\normalsize}{\thesection. }{2pt}{\normalsize}

%Definition of subsection format
%\titleformat{\subsection}{\bfseries\normalsize}{\thesubsection.}{2pt}{\normalsize}

% needed for splitting big tables across pages
\usepackage{longtable} 

%claims to produce more beautiful tables that more closely resemble those used in traditional typesetting, for \toprule, \bottomrule
\usepackage{booktabs}

% for references 
\usepackage{cite}

%In oreder to do the citation of the references as lilnks
\usepackage{hyperref}

%Packages for tables
\usepackage{array}		%We use it in order to have the m arg, insde the table env. Put the text in the middle of the cel
\usepackage{multirow} 	% Create tabular cells spanning multiple rows.

%This is the way to define the motive for the definitions
\usepackage{amsthm}  %provides an enhanced version of LATEX’s \newtheorem command
%plain, Sets the styling for the numbered environment 
\theoremstyle{definition}	
%First argument is  the name of the Env=definition, second the word that will be printed. Third [section]  restarts the definition counter at every new section.
\newtheorem{definition}{Definition}[section] 


% Required package for mathematical facilities
\usepackage{amsmath}

%Required package in oreder to use the [ noitemsep , nolistsep ] in the lists env, in oreder to remove the space between lines of the list.
\usepackage{enumitem}

%We use these packages in order to add the python code inside the document.
\usepackage{listings} 
\usepackage{xcolor}
\definecolor{codegreen}{rgb}{0,0.6,0}
\definecolor{codegray}{rgb}{0.5,0.5,0.5}
\definecolor{codepurple}{rgb}{0.58,0,0.82}
\definecolor{backcolour}{rgb}{0.95,0.95,0.92}

\lstdefinestyle{mystyle}{
    backgroundcolor=\color{backcolour},   
    commentstyle=\color{codegreen},
    keywordstyle=\color{magenta},
    numberstyle=\tiny\color{codegray},
    stringstyle=\color{codepurple},
    basicstyle=\ttfamily\footnotesize,
    breakatwhitespace=false,         
    breaklines=true,                 
    captionpos=b,                    
    keepspaces=true,                 
    numbers=left,                    
    numbersep=5pt,                  
    showspaces=false,                
    showstringspaces=false,
    showtabs=false,                  
    tabsize=2
}

\lstset{style=mystyle}


%==========================================================================================================================
%==========================================================================================================================

\begin{document}		 % Beginning of the actual document.
\pagestyle{plain}	 % Sets the style of the current page, and all subsequent pages, to ⟨plain⟩

%=======================================================================
%====================================  FrontMatter ========================== ========================================================================
	
	% -Cover English
	% -Blank Page
	% -Cover Greek
	% -Blank Page
	% -Exam Committee Enlglish
	% -Blank Page
	% -Exam Committee Greek
	% -Blank Page
	% -Abstract
	% -Blank Page
	% -Περίληψη
	% -Blank Page
	% -Inscriptions (Αφιερώσεις)
	% -Blank Page
	% -Acknowledgments
	% -Blank Page
	% -Table of Contents
	% -List of Figures
	% -Blank Page
	% -List of Tables
	% -Blank Page


	% No numbers on the following Pages
	\pagenumbering{gobble}	
	
% FIXME
%\begin{titlepage} ---If I added it, I can ommit the above page numbering
	
		
	%%%%%%%%%%%%%%%%%%%%%%%%%%%%%%%%%%%%%%%%%%%%%%%%%%%%%%%%%%%%%%%%%%%%%%%%%
%
% Cover page English version
%
%%%%%%%%%%%%%%%%%%%%%%%%%%%%%%%%%%%%%%%%%%%%%%%%%%%%%%%%%%%%%%%%%%%%%%%%%

\begin{center}
	\includegraphics{Emblems/athena-black}
\end{center}

\begin{minipage}[t]{\textwidth}

	\begin{center}
	
	  	{\large \bfseries 
	  		NATIONAL AND KAPODISTRIAN UNIVERSITY OF ATHENS 
	  	}
      	\linebreak

		{\bfseries
			SCHOOL OF SCIENCES \\ DEPARTMENT OF INFORMATICS AND TELECOMMUNICATIONS 
	 	}
		\linebreak	

		{\bfseries
      		THEORETICAL INFORMATICS
      	}
      	\linebreak\linebreak\linebreak
      	\linebreak\linebreak\linebreak	
      	    	
      	{\bfseries
     		MSc THESIS
		}
      	\linebreak\linebreak\linebreak
	
		{\Large \bfseries
     		Generative AI in Graphics
 		}	
      	\linebreak\linebreak\linebreak	
	
		{\bfseries
	  		Evgenios N. Mazarakis
	  	}		
	
	\end{center}

\end{minipage}


\vfill
{\bfseries Supervisors:}
	\begin{minipage}[t]{\textwidth}
        {\bfseries Theocharis Theocharis}, Professor NKUA\\
    		{\bfseries \textcolor{red}{TODO} }, Professor NKUA
    \end{minipage}
    \\


\begin{center}
    {\bfseries 
    		ATHENS 
    	}
    \linebreak

    {\bfseries 
    		DECEMBER 2023 
    	}
\end{center}

\clearpage

%\newpage\null\newpage




	%Blank Page of my Master Thesis with arial 

% Create a paragraph consisting of lines that are centered within the left and right margins on the current page.
\begin{center}
	\newpage
\end{center}	

	%%%%%%%%%%%%%%%%%%%%%%%%%%%%%%%%%%%%%%%%%%%%%%%%%%%%%%%%%%%%%%%%%%%%%%%%%
%
% Cover page Greek version
%
%%%%%%%%%%%%%%%%%%%%%%%%%%%%%%%%%%%%%%%%%%%%%%%%%%%%%%%%%%%%%%%%%%%%%%%%%

\begin{center}
	\includegraphics{Emblems/athena-black}
\end{center}

\begin{minipage}[t]{\textwidth}

	\begin{center}
	
	  	{\large \bfseries 
	  		ΕΘΝΙΚΟ ΚΑΙ ΚΑΠΟΔΙΣΤΡΙΑΚΟ ΠΑΝΕΠΙΣΤΗΜΙΟ ΑΘΗΝΩΝ
	  	}
      	\linebreak

		{\bfseries
			ΣΧΟΛΗ ΘΕΤΙΚΩΝ ΕΠΙΣΤΗΜΩΝ \\ ΤΜΗΜΑ ΠΛΗΡΟΦΟΡΙΚΗΣ ΚΑΙ ΤΗΛΕΠΙΚΟΙΝΩΝΙΩΝ
	 	}
		\linebreak	

		{\bfseries
      		ΘΕΩΡΗΤΙΚΗ ΠΛΗΡΟΦΟΡΙΚΗ
      	}
      	\linebreak\linebreak\linebreak
      	\linebreak\linebreak\linebreak	
      	    	
      	{\bfseries
     		ΔΙΠΛΩΜΑΤΙΚΗ ΕΡΓΑΣΙΑ
		}
      	\linebreak\linebreak\linebreak
	
		{\Large \bfseries
     		Generative AI in Graphics
 		}	
      	\linebreak\linebreak\linebreak	
	
		{\bfseries
	  		Ευγένιος Ν. Μαζαράκης
	  	}		
	
	\end{center}

\end{minipage}


\vfill
{\bfseries Επιβλέποντες:}
	\begin{minipage}[t]{\textwidth}
        {\bfseries Θεοχάρης Θεοχάρης}, Καθηγητής ΕΚΠΑ\\
    		{\bfseries \textcolor{red}{TODO} }, Καθηγητής ΕΚΠΑ
    \end{minipage}
    \\


\begin{center}
    {\bfseries 
    		ΑΘΗΜΑ
    	}
    \linebreak

    {\bfseries 
    		ΔΕΚΕΜΒΡΙΟΣ 2023 
    	}
\end{center}

\clearpage

%\newpage\null\newpage



	
	%Blank Page of my Master Thesis with arial 

% Create a paragraph consisting of lines that are centered within the left and right margins on the current page.
\begin{center}
	\newpage
\end{center}	

	%%%%%%%%%%%%%%%%%%%%%%%%%%%%%%%%%%%%%%%%%%%%%%%%%%%%%%%%%%%%%%%%%%%%%%%%%
%
% Exam Committee page English version
%
%%%%%%%%%%%%%%%%%%%%%%%%%%%%%%%%%%%%%%%%%%%%%%%%%%%%%%%%%%%%%%%%%%%%%%%%%

\begin{center}
	{\bfseries
		MSc THESIS
	}
	\linebreak
	
	%Title of the Master Thesis
	{ Generative AI in Graphics }
	\linebreak

	{\bfseries
		Evgenios N. Mazarakis\\
    		S.N.: 
	}
	{M1458}
	\linebreak\linebreak
\end{center}


\vfill
{\bfseries \MakeUppercase {Supervisors}:}
	\begin{minipage}[t]{\textwidth}
        {\bfseries Theocharis Theocharis}, Professor NKUA\\
    		{\bfseries \textcolor{red}{TODO} }, Professor NKUA
    \end{minipage}
    \\




%\linebreak\linebreak
%\linebreak

\vfill
{
	{\bfseries EXAMINATION COMMITTEE:}
    \begin{minipage}[t]{\textwidth}
        {\bfseries \textcolor{red}{TODO} }, Professor NKUA\\
        {\bfseries \textcolor{red}{TODO} }, Professor NKUA\\
        {\bfseries \textcolor{red}{TODO} }, Professor NKUA
    \end{minipage}
}


\vfill
\begin{center}
	{\bfseries
		Examination Date: \textcolor{red}{DD Month Year} 
	}
\end{center}
\clearpage
%\newpage\null\newpage	
	%Blank Page of my Master Thesis with arial 

% Create a paragraph consisting of lines that are centered within the left and right margins on the current page.
\begin{center}
	\newpage
\end{center}	
	
	
	%%%%%%%%%%%%%%%%%%%%%%%%%%%%%%%%%%%%%%%%%%%%%%%%%%%%%%%%%%%%%%%%%%%%%%%%%
%
% Exam Committee page  Greek version
%
%%%%%%%%%%%%%%%%%%%%%%%%%%%%%%%%%%%%%%%%%%%%%%%%%%%%%%%%%%%%%%%%%%%%%%%%%

\begin{center}
	{\bfseries
		ΔΙΠΛΩΜΑΤΙΚΗ ΕΡΓΑΣΙΑ
	}
	\linebreak
	
	%Title of the Master Thesis
	{ Generative AI in Graphics }
	\linebreak

	{\bfseries
		Ευγένιος Ν. Μαζαράκης\\
    		Α.Μ.: 
	}
	{M1458}
	\linebreak\linebreak
\end{center}


\vfill
{\bfseries \MakeUppercase {ΕΠΙΒΛΕΠΟΝΤΕΣ}:}
	\begin{minipage}[t]{\textwidth}
        {\bfseries Θεοχάρης Θεοχάρης}, Καθηγητής ΕΚΠΑ\\
    		{\bfseries \textcolor{red}{TODO} }, Καθηγητής ΕΚΠΑ
    \end{minipage}
    \\


%\linebreak\linebreak
%\linebreak

\vfill
{
	{\bfseries ΕΞΕΤΑΣΤΙΚΗ ΕΠΙΤΡΟΠΗ:}
    \begin{minipage}[t]{\textwidth}
        {\bfseries \textcolor{red}{TODO} }, Καθηγητής ΕΚΠΑ\\
        {\bfseries \textcolor{red}{TODO} }, Καθηγητής ΕΚΠΑ\\
        {\bfseries \textcolor{red}{TODO} }, Καθηγητής ΕΚΠΑ
    \end{minipage}
}


\vfill
\begin{center}
	{\bfseries
		Ημερομηνία Εξέτασης: \textcolor{red}{DD Month Year} 
	}
\end{center}
\clearpage
%\newpage\null\newpage
	%Blank Page of my Master Thesis with arial 

% Create a paragraph consisting of lines that are centered within the left and right margins on the current page.
\begin{center}
	\newpage
\end{center}	
	
	%%%%%%%%%%%%%%%%%%%%%%%%%%%%%%%%%%%%%%%%%%%%%%%%%%%%%%%%%%%%%%%%%%%%%%%%%
%
% Abstract  page  English version
%
%%%%%%%%%%%%%%%%%%%%%%%%%%%%%%%%%%%%%%%%%%%%%%%%%%%%%%%%%%%%%%%%%%%%%%%%%
\thispagestyle{empty}
\chapter*{ {ABSTRACT} }
%\setmainlanguage{english}


\setlength{\parindent}{0pt}	%Για να μην έχει εσοχή η κάθε παραγράφος 

This master's thesis delves into the cutting-edge domain of Generative Artificial Intelligence (AI) with a primary focus on graphics applications. The study explores various deep learning techniques, including Variational Autoencoders (VAEs), Generative Adversarial Networks (GANs), Diffusion Models, and Transformers. These methods have revolutionized the field by enabling the generation of realistic and high-quality content, ranging from images to entire scenes.\\

The investigation begins with an in-depth exploration of Variational Autoencoders, emphasizing their role in capturing latent representations of data. Subsequently, the research transitions to Generative Adversarial Networks, discussing their adversarial training process for generating authentic content. Additionally, the study reviews Diffusion Models, which excel in probabilistic generative modeling, and Transformers, renowned for their success in sequential data generation tasks.\\

A significant portion of the thesis is dedicated to the execution of three models in the graphics domain based on some parameters. Executing generative models through Python code or run them locally through an interface empowers users to harness advanced AI techniques. This enables the creation of diverse, high-quality content, revolutionizing applications in various domains seamlessly.\\

The findings highlight the transformative impact of these generative models on the field of graphics, showcasing their ability to create immersive and novel visual content through the amalgamation of sophisticated deep learning techniques.

  
\vfill

{\bfseries SUBJECT AREA}: Generative AI\\

{\bfseries KEYWORDS}: Generative AI, Artificial Intelligence, Models, Graphics, Deep Learning


%%%%%%%%%%%%%%%%%%%%%%%%%%%%%%%%%%%%%%%%%%

	%Blank Page of my Master Thesis with arial 

% Create a paragraph consisting of lines that are centered within the left and right margins on the current page.
\begin{center}
	\newpage
\end{center}	
		
	%%%%%%%%%%%%%%%%%%%%%%%%%%%%%%%%%%%%%%%%%%%%%%%%%%%%%%%%%%%%%%%%%%%%%%%%%
%
% Abstract page Greek version
%
%%%%%%%%%%%%%%%%%%%%%%%%%%%%%%%%%%%%%%%%%%%%%%%%%%%%%%%%%%%%%%%%%%%%%%%%%

\thispagestyle{empty}
\chapter*{ {ΠΕΡΙΛΗΨΗ} }
%\setmainlanguage{english}

%Για να μην έχει εσοχή η κάθε παραγράφος 
\setlength{\parindent}{0pt}	

%\emergencystretch is a TeX-internal “dimen” register, and can be set as normal for dimens in Plain TeX; in LaTeX
\setlength{\emergencystretch}{3em} 

H πτυχιακή εργασία εμβαθύνει στον τομέα αιχμής της Generative Artificial Intelligence (AI) εστιάζοντας στις εφαρμογές των γραφικών. 
Η μελέτη διερευνά διάφορες deep learning τεχνικές, συμπεριλαμβανομένων των Variational Autoencoders (VAEs), των Generative Adversarial Networks (GANs), των Diffusion Models και των Transformers. Αυτές οι μέθοδοι έχουν φέρει επανάσταση στον τομέα της Τεχνητής Νοημοσύνης, επιτρέποντας τη δημιουργία ρεαλιστικού και υψηλής ποιότητας περιεχομένου, που κυμαίνεται από εικόνες έως ολόκληρες σκηνές.\\

Η διπλωματική εργασία ξεκινά με μια εις βάθος εξερεύνηση των Variational Autoencoder, δίνοντας έμφαση στον ρόλο τους στην  λανθάνουσα αναπαράσταση των δεδομένων. Στη συνέχεια, η έρευνα μεταβαίνει στα Generative Adversarial Networks, αναδεικνύοντας τη διαδικασία εκπαίδευσης των συστατικών τους στοιχείων για τη δημιουργία αυθεντικού περιεχομένου. Επιπλέον, η μελέτη εξετάζει τα Diffusion Models, τα οποία διαπρέπουν στην probabilistic generative μοντελοποίηση, και τους Transformers , γνωστούς για την επιτυχία τους σε εργασίες παραγωγής διαδοχικών δεδομένων (sequential data).\\

Ένα σημαντικό μέρος της διπλωματικής εργασίας είναι αφιερωμένο στην εκτέλεση τριών μοντέλων στον τομέα των γραφικών με βάση ορισμένες παραμέτρους. Η εκτέλεση των μοντέλων μέσω κώδικα Python ή εκτελώντας τα τοπικά μέσω μιας διεπαφής   δίνει τη δυνατότητα στους χρήστες να αξιοποιήσουν προηγμένες τεχνικές AI. Αυτό επιτρέπει τη δημιουργία ποικίλου περιεχομένου υψηλής ποιότητας, φέρνοντας επανάσταση στις εφαρμογές σε διάφορους τομείς απρόσκοπτα.\\

Τα ευρήματα υπογραμμίζουν την επίδραση αυτών των παραγωγικών μοντέλων στο πεδίο των γραφικών, επιδεικνύοντας την ικανότητά τους να δημιουργούν πρωτότυπο και νέο οπτικό περιεχόμενο μέσω της συγχώνευσης εξελιγμένων τεχνικών βαθιάς μάθησης.

  
\vfill

{\bfseries ΘΕΜΑΤΙΚΗ ΠΕΡΙΟΧΗ}: Γενεσιουργός Τεχνητή Νοημοσύνη \\

{\bfseries ΛΕΞΕΙΣ ΚΛΙΕΔΙΑ}: Γενεσιουργός Τεχνητή Νοημοσύνη, Τεχνητή Νοημοσύνη, Μοντέλα, Γραφικά, Βαθιά Μάθηση


%%%%%%%%%%%%%%%%%%%%%%%%%%%%%%%%%%%%%%%%%%


	%Blank Page of my Master Thesis with arial 

% Create a paragraph consisting of lines that are centered within the left and right margins on the current page.
\begin{center}
	\newpage
\end{center}	
	
	%%%%%%%%%%%%%%%%%%%%%%%%%%%%%%%%%%%%%%%%
%
% Inscription page English Version
%
%%%%%%%%%%%%%%%%%%%%%%%%%%%%%%%%%%%%%%%%

%\chapter*{Αφιέρωση}
%\chapter*{ \centering {\normalsize {ΑΦΙΕΡΩΣΗ} } }


\thispagestyle{empty}
\cleardoublepage
%\vspace{1cm}

\begin{flushright}

	\hspace{2cm}
	\vspace{4cm}

	\textit{	Η διπλωματική εργασία\\
				είναι αφιερωμένη \\
				στο  Νίκο  Μαζαράκη,\\
				στην Έλλη  Μαζαράκη,\\
				και στο  Θεόφιλο Μαζαράκη.\\}
\end{flushright}
\clearpage

	%Blank Page of my Master Thesis with arial 

% Create a paragraph consisting of lines that are centered within the left and right margins on the current page.
\begin{center}
	\newpage
\end{center}	
		
	%%%%%%%%%%%%%%%%%%%%%%%%%%%%%%%%%%%%%%%%%%%%%%%%%%%%%%%%%%%%%%%%%%%%%%%%%
%
% Acknowledgements page  English version
%
%%%%%%%%%%%%%%%%%%%%%%%%%%%%%%%%%%%%%%%%%%%%%%%%%%%%%%%%%%%%%%%%%%%%%%%%%

\chapter*{ {ACKNOWELEDGMENTS} }
\thispagestyle{empty}
%\setmainlanguage{english}

{
I feel the need to express my immense gratitude to my parents Nico and Elli Mazarakis and to my beloved brother Theofilos Mazarakis, for everything they have offered me during my student years as well for their undivided support in my every choice. At this point, I would like to thank my supervisor Mr. Theoharis Theoharis once again for trusting and devoting valuable time to get the job done.
}
  
	%Blank Page of my Master Thesis with arial 

% Create a paragraph consisting of lines that are centered within the left and right margins on the current page.
\begin{center}
	\newpage
\end{center}	
		
		
			
%============================================================================
% ===================== Table of Contents, List of Figures, List of Tables =========================
%============================================================================

	
	% -Πίνακας Περιεχομένων	
	\renewcommand\contentsname{CONTENTS}
	\tableofcontents{\thispagestyle{empty}}		
	\addtocontents{toc}{~\hfill\textbf{Page}\par}	

	
	% -Κατάλογος Εικόνων
	\renewcommand\listfigurename{LIST OF FIGURES}
	\listoffigures{\thispagestyle{empty}}	
	
	%Blank Page of my Master Thesis with arial 

% Create a paragraph consisting of lines that are centered within the left and right margins on the current page.
\begin{center}
	\newpage
\end{center}		
	
	% -Κατάλογος Πινάκων
	\renewcommand\listtablename{LIST OF TABLES}	
	\listoftables{\thispagestyle{empty}}
	
	%Blank Page of my Master Thesis with arial 

% Create a paragraph consisting of lines that are centered within the left and right margins on the current page.
\begin{center}
	\newpage
\end{center}	

%\end{titlepage}
	
	% -Αρχικοποιώ Τη Σελίδα
	\cleardoublepage
	\pagenumbering{arabic}
	\setcounter{page}{8}
	
	%Για να μην έχει εσοχή η κάθε παραγράφος 
	\setlength{\parindent}{0pt}		


%============================================================================
%=====================================  MainMatter  =============================== %=============================================================================


	% -Preface		- In English
	% -Πρόλογος     - In Greek
	% -Chapter 1	    - Introduction
	% -Chapter 2 	- Running Models
	% -Chapter 3 	- Conclusions

	\chapter*{PREFACE}
		\input{MainMatter/PrefaceEn/PrefaceEn}
		
	%Blank Page of my Master Thesis with arial 

% Create a paragraph consisting of lines that are centered within the left and right margins on the current page.
\begin{center}
	\newpage
\end{center}	
		
	\chapter*{ΠΡΟΛΟΓΟΣ}		% If you want to disappear it from the ToC, go to the tex file and commented the \addcontentsline.
		\input{MainMatter/PrefaceGr/PrefaceGr}
	
	%Blank Page of my Master Thesis with arial 

% Create a paragraph consisting of lines that are centered within the left and right margins on the current page.
\begin{center}
	\newpage
\end{center}	
	
	%\cleardoublepage
	%\pagestyle{fancy}

	
	\chapter{INTRODUCTION}
	
		The thesis is structured as follows: The first chapter provides comprehensive definitions within the domain of Generative AI and presents a taxonomy of AI models categorized by their generated content. The second chapter introduces three models, which were executed based on specified parameters such as prompt, width, and height. Additionally, this chapter presents the benchmark utilized for human evaluation of the models. Finally, the third chapter discusses the evaluation results and highlights findings derived from model executions. The structure of the master thesis is given in the following figure.
		\begin{figure}[h!]	%h!
				\begin{center}	
	 				%scale the image 0.5 of its real size. Instead of scale we can use width=5cm, height=4cm.
					\includegraphics[scale=0.5]{Figures/000.Thesis_Organization}	
					\caption{Organization of this thesis.}
				\end{center}
		\end{figure}		
		
		
		
			\section{The age of generative AI​}	
					\input{MainMatter/Introduction/theAgeOfGenerativeAI/theAgeOfGenerativeAI}
			
			% or Taxanomy of the generative AI models
			\section{Classification of generative AI models}
				\input{MainMatter/Introduction/literarureReview/ClassificationOfGenModels}



	\chapter{EVALUATION}
	We perform a human evaluation comparing the most common open-source (Stable Diffusion, Crayon(DALLE mini) ),  and commercial (DALL-E 2) models. These models have 3 common hyperparameters: prompt, height, width. For the evaluation of these three models, we provide a benchmark of seven task-types with twelve prompts of each task type.
	
		\section{Models​}	
					\input{MainMatter/Evaluation/models/models}
					
		\section{Benchmark}	
					\input{MainMatter/Evaluation/benchmark/benchmark}
	

	\chapter{CONCLUSION}
	The quantitative method for evaluating text-to-image generative models, using a benchmark that encompasses various model competencies and applications, provides us with insights into the capabilities of each model. Therefore, understanding the limitations is critical for picking the suitable model for each task and application, advancing the quality of generative models, and aligning their performance with human goals.\\
	
			\section{Results}	
					\input{MainMatter/Conclusion/results/results}
		
		\section{Findings}	
					\input{MainMatter/Conclusion/findings/findings}
			


%============================================================================
%======================== BackMatter  ============================================
%============================================================================

	%		-Glossary
	% 	-Abbreviations/Acronyms
	% 	-References
		
		
		%\include{BackMatter/Glossary/Glossary}
		
		%Blank Page of my Master Thesis with arial 

% Create a paragraph consisting of lines that are centered within the left and right margins on the current page.
\begin{center}
	\newpage
\end{center}	
		%Akronimia of my thesis with Arial

%Syntomografies
%\printnoidxglossary[type=sint,sort=use]
%\addcontentsline{toc}{chapter}{Συντομογραφίες}


\chapter*{\centering{ABBREVIATIONS - ACRONYMS}}
\addcontentsline{toc}{chapter}{ABBREVIATIONS - ACRONYMS}	% Add Acronyms to the ToC
\begin{center}

    %\renewcommand{\arraystretch}{1.5}
   \begin{longtable}{ l @{\qquad} l }    %\begin{longtable}{p{3 cm}p{6 cm}}
       \toprule
        AI    		& Artificial Intelligence   						 	 \\
		D 			& Discriminator									 	 \\ 
        DNN 		& Deep Neural Network						 	 \\
        DL 		& Deep Learning								 	 \\
        DDPMs  & Denoising Diffusion Probabilistic Models \\        
    	FM		& Flow Matching									 \\
        GPT		& Generative Pre-trained Transformer	 \\
        GAI 		& Generative  Artificial Intelligence  		 \\
        GANs 	& Generative Adversarial Networks 		 \\ 
        G 			& Generator										\\
 		LLM 		& Large Language Model						 \\        
        ML   		& Machine Learning							\\ 
        NL 		& Natural Language							\\
        NN   		& Neural Network 								 \\
		PL 		& Programming Language				     \\
		SSR 		& Spatial Super-Resolution					 \\
		SGMs    & Score-Based Generative Models 		\\	
		SDEs     & Stochastic Differential Equations		\\
		TSR 		& Temporal Super-Resolution				 \\
		T2V 		& Text-to-Video									 \\
		T2I 		& Text-to-Image									\\
        TTS 		& Text to Speech								\\   
        VAEs 	& Variational Auto-Encoders 				 \\
	    2D 		& 2 Dimensions									 \\
        3D 		& 3 Dimensions									 \\        
        prt 		& prompt									         \\    
        SD		& Stable Diffusion    							\\
        DL			& DALLE-2											\\
        CR		& Crayon											\\
       \bottomrule
    \end{longtable}

\end{center}



		%Blank Page of my Master Thesis with arial 

% Create a paragraph consisting of lines that are centered within the left and right margins on the current page.
\begin{center}
	\newpage
\end{center}	
		

	  \addcontentsline{toc}{chapter}{REFERENCES}		% Add references to the ToC
	  \bibliographystyle{plain}
	  \renewcommand\bibname{REFERENCES}			   %Change the name, from Bibliography to References
	  \bibliography{BackMatter/References/References}	
	 
	 

\end{document}	% End of the actual document.